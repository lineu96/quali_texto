\chapter{Revisão de litaratura}

\label{cap:literatura}

Nossa revisão de literatura aborda predominantemente três temas. O primeiro deles é uma revisão da estrutura geral e estimação dos parâmetros de um modelo multivariado de covariância linear generalizada, baseado nas ideias de \citet{Bonat16}. A segunda parte da revisão de literatura diz respeito ao procedimento chamado teste de hipóteses com o foco de tratar do objetivo, notação, componentes, aplicação deste tipo de procedimento no contexto de modelos de regressão e ainda quais os testes mais comuns. Por fim, a última parte da revisão diz respeito às análises de variância, que podem ser vistos como procedimentos baseados em testes de hipóteses sequenciais para avaliar os parâmetros de um modelo de regressão.

\section{Modelos Multivariados de Covariância Linear Generalizada}

Os Modelos Linerares Generalizados (GLM), propostos por \citet{Nelder72}, são uma forma de modelagem que lida exclusivamente com uma resposta para dados de diferentes naturezas, tais como respostas contínuas, binárias e contagens. Tais características tornam essa classe de modelos uma flexível ferramenta de modelagem aplicável a diversos tipos de problemas. Contudo, por mais flexível e discutida na literatura, essa classe apresenta ao menos três importantes restrições: um leque restrito de distribuições disponíveis para modelagem, a incapacidade de lidar com observações dependentes e a incapacidade de lidar com múltiplas respostas simultaneamente. 

Com o objetivo de contornar estas restrições, foi proposta por \citet{Bonat16}, uma estrutura geral para análise de dados não gaussianos com múltiplas respostas em que não se faz suposições quanto à independência das observações: os chamados Modelos Multivariados de Covariância Linear Generalizada (McGLMs). Tais modelos, levam em conta a não normalidade por meio de uma função de variância. Além disso, a estrutura média é modelada por meio de uma função de ligação e um preditor linear. Os parâmetros dos modelos são obtidos através de funções de estimação baseadas em suposições de segundo momento.

Vamos discutir os McGLMs como uma extensão dos GLMs. Vale ressaltar que é usada uma especificação menos usual de um Modelo Linear Generalizado, porém trata-se de uma notação mais conveniente para chegar à uma especificação mais simples de um Modelo Multivariado de Covariância Linear Generalizada.

\subsection{Modelo linear generalizado}

Para definição da extensão de um modelo linear generalizdo (GLM descrita em \citet{Bonat16}, considere $\boldsymbol{Y}$ um vetor $N \times 1$ de valores observados da variável resposta, $\boldsymbol{X}$ uma matriz de delineamento $N \times k$ e $\boldsymbol{\beta}$ um vetor de parâmetros de regressão $k \times 1$. Com isso, um GLM pode ser escrito da seguinte forma 

\begin{equation}
\label{eq:glm}
      \begin{aligned}
        \mathrm{E}(\boldsymbol{Y}) &=
         \boldsymbol{\mu} =
            g^{-1}(\boldsymbol{X} \boldsymbol{\beta}),
            \\
        \mathrm{Var}(\boldsymbol{Y}) &=
          \Sigma =
          \mathrm{V}\left(\boldsymbol{\mu}; p\right)^{1/2}\left(\tau_0\boldsymbol{I}\right)\mathrm{V}\left(\boldsymbol{\mu}; p\right)^{1/2},
      \end{aligned}
\end{equation}

\noindent em que $g(.)$ é a função de ligação, $\mathrm{V}\left(\boldsymbol{\mu}; p\right)$ é uma matriz diagonal em que as entradas principais são dadas pela função de variância aplicada ao vetor $\boldsymbol{\mu}$, $p$ é o parâmetro de potência, $\tau_0$ o parâmetro de dispersão e $\boldsymbol{I}$ é a matriz identidade de ordem $N\times N$.

Nesta extensão, os GLMs fazem uso de apenas duas funções, a função de variância e de ligação. Diferentes escolhas de funções de variância implicam em diferentes suposições a respeito da distribuição da variável resposta. Dentre as funções de variância conhecidas, podemos citar:

1. A função de variância potência, que caracteriza a família Tweedie de distribuições, em que a função de variância é dada por $\vartheta\left(\boldsymbol{\mu}; p\right) = \mu^p$, na qual destacam-se as distribuições: normal ($p$ = 0), Poisson ($p$ = 1), gama ($p$ = 2) e  normal inversa ($p$ = 3). Para mais informações consulte \citet{Jorgensen87} e \citet{Jorgensen97}.

2. A função de dispersão Poisson–Tweedie, a qual caracteriza a família Poisson-Tweedie de distribuições, que visa contornar a inflexibilidade da utilização da função de variância potência para  respostas discretas. A família Poisson-Tweedie tem função de dispersão dada por $\vartheta\left(\boldsymbol{\mu}; p\right) = \mu + \mu^p$ e tem como casos particulares os mais famosos modelos para dados de contagem: Hermite ($p$ = 0), Neyman tipo A ($p$ = 1), binomial negativa ($p$ = 2) e Poisson–inversa gaussiana (p = $3$) \citep{Jorgensen15}. Não se trata de uma função de variância usual, mas é uma função que caracteriza o relacionamento entre média e variância.

3. A função de variância binomial, dada por $\vartheta(\boldsymbol{\mu}) = \mu(1 - \mu)$, utilizada quando a variável resposta é binária, restrita a um intervalo ou quando tem-se o  número de sucessos em um número de tentativas.

Lembre-se que o GLM é uma classe de modelos de regressão univariados em que um dos pressupostos é a independência entre as observações. Esta independência é especificada na matriz identidade no centro \autoref{eq:glm}. Podemos imaginar que, substituindo esta matriz identidade por uma matriz qualquer que reflita a relação entre os indivíduos da amostra teremos uma extensão do Modelo Linear Generalizado para observações dependentes. É justamente essa a ideia dos Modelos de Covariância Linear Generalizada, o cGLM.

\subsection{Modelo de covariância linear generalizada}

Os modelos de covariância linear generalizada (cGLM) são uma alternativa para problemas em que a suposição de independência entre as observações não é atendida. Neste caso, a solução proposta é substituir a matriz identidade $\boldsymbol{I}$ da \autoref{eq:glm} por uma matriz não diagonal $\boldsymbol{\Omega({\tau})}$ que descreva adequadamente a estrutura de correlação entre as observações. Trata-se de uma ideia similar à proposta de \citet{Liang86} nos modelos GEE (Equações de Estimação Generalizadas), em que utiliza-se uma matriz de correlação de trabalho para considerar a dependência entre as observações. A matriz $\boldsymbol{\Omega({\tau})}$ é descrita como uma combinação de matrizes conhecidas tal como nas propostas de \citet{Anderson73} e \citet{Pourahmadi00}, podendo ser escrita da forma

\begin{equation}
\label{eq:cov}
h\left \{ \boldsymbol{\Omega}(\boldsymbol{\tau}) \right \} = \tau_0Z_0 + \ldots + \tau_DZ_D,
\end{equation}

\noindent em que $h(.)$ é a função de ligação de covariância, $Z_d$ com $d$ = 0,$\ldots$, D são matrizes que representam a estrutura de covariância presente nos dados e $\boldsymbol{\tau}$ = $(\tau_0, \ldots, \tau_D)$ é um vetor $(D + 1) \times 1$ de parâmetros de dispersão. Tal estrutura pode ser vista como um análogo ao preditor linear para a média e foi nomeado como preditor linear matricial, a especificação da função de ligação de covariância é discutida por \citet{Pinheiro96}. É possível selecionar combinações de matrizes para se obter os mais conhecidos modelos da literatura para dados longitudinais, séries temporais, dados espaciais e espaço-temporais. Maiores detalhes são discutidos por \citet{Demidenko13}.

Com isso, substituindo a matriz identidade da \autoref{eq:glm} pela \autoref{eq:cov}, temos uma classe com toda a flexibilidade dos GLMs, porém contornando a restrição da independência entre as observações desde que o preditor linear matricial seja adequadamente especificado. Deste modo, é contornada a restrição da incapacidade de lidar com observações dependentes. Outra restrição diz respeito às múltiplas respostas e, contornando este problema, chegamos ao McGLM.

\subsection{Modelos multivariados de covariância linear generalizada}

Os modelos multivariados de covariância linear generalizada (McGLMs) podem ser entendidos como uma extensão multivariada dos cGLMs que contornam as principais restrições presentes nos GLMs. Para definição de um McGLM considre $\boldsymbol{Y}_{N \times R} = \left \{ \boldsymbol{Y}_1, \dots, \boldsymbol{Y}_R \right \}$ uma  matriz de variáveis resposta e $\boldsymbol{M}_{N \times R} = \left \{ \boldsymbol{\mu}_1, \dots, \boldsymbol{\mu}_R \right \}$ uma matriz de valores esperados. Cada uma das variáveis resposta tem sua própria matriz de variância e covariância, responsável por modelar a covariância dentro de cada resposta, sendo expressa por

\begin{equation}
\Sigma_r =
\mathrm{V}_r\left(\boldsymbol{\mu}_r; p\right)^{1/2}\boldsymbol{\Omega}_r\left(\boldsymbol{\tau}\right)\mathrm{V}_r\left(\boldsymbol{\mu}_r; p\right)^{1/2}.
\end{equation}

Além disso, é necessária uma matriz de correlação $\Sigma_b$, de ordem $R \times R$, que descreve a correlação entre as variáveis resposta. Para a especificação da matriz de variância e covariância conjunta é utilizado o produto Kronecker generalizado, proposto por \citet{martinez13}.

Finalmente, um MCGLM é descrito como

\begin{equation}
\label{eq:mcglm}
      \begin{aligned}
        \mathrm{E}(\boldsymbol{Y}) &=
          \boldsymbol{M} =
            \{g_1^{-1}(\boldsymbol{X}_1 \boldsymbol{\beta}_1),
            \ldots,
            g_R^{-1}(\boldsymbol{X}_R \boldsymbol{\beta}_R)\}
          \\
        \mathrm{Var}(\boldsymbol{Y}) &=
          \boldsymbol{C} =
            \boldsymbol{\Sigma}_R \overset{G} \otimes
            \boldsymbol{\Sigma}_b,
      \end{aligned}
\end{equation}

\noindent em que $\boldsymbol{\Sigma}_R \overset{G} \otimes \boldsymbol{\Sigma}_b = \mathrm{Bdiag}(\tilde{\boldsymbol{\Sigma}}_1, \ldots, \tilde{\boldsymbol{\Sigma}}_R) (\boldsymbol{\Sigma}_b \otimes \boldsymbol{I}) \mathrm{Bdiag}(\tilde{\boldsymbol{\Sigma}}_1^\top, \ldots, \tilde{\boldsymbol{\Sigma}}_R^\top)$ é o produto generalizado de Kronecker, a matriz $\tilde{\boldsymbol{\Sigma}}_r$ denota a matriz triangular inferior da decomposição de Cholesky da matriz ${\boldsymbol{\Sigma}}_r$, o operador $\mathrm{Bdiag}$ denota a matriz bloco-diagonal e $\boldsymbol{I}$ uma matriz identidade $N \times N$. Com isso, chega-se a uma classe de modelos na qual através da especificação da função de variâcia têm-se um leque maior de distribuições disponíveis, através do preditor matricial se torna possível a modelagem de dados com estrutura de covariância, e ainda é possível a modelagem de múltiplas respostas.

\subsection{Estimação e inferência}

Os McGLMs são ajustados baseados no método de funções de estimação descritos em detalhes por \citet{Bonat16} e \citet{jorg04}. Nesta seção é apresentada uma visão geral do algoritmo e da distribuição assintótica dos estimadores baseados em funções de estimação.

As suposições de segundo momento dos McGLM permitem a divisão dos
parâmetros em dois conjuntos: $\boldsymbol{\theta} = (\boldsymbol{\beta}^{\top}, \boldsymbol{\lambda}^{\top})^{\top}$. Desta forma, $\boldsymbol{\beta} = (\boldsymbol{\beta}_1^\top, \ldots, \boldsymbol{\beta}_R^\top)^\top$ é um vetor $K \times 1$ de parâmetros de regressão e $\boldsymbol{\lambda} = (\rho_1, \ldots, \rho_{R(R-1)/2}, p_1, \ldots, p_R, \boldsymbol{\tau}_1^\top, \ldots, \boldsymbol{\tau}_R^\top)^\top$ é um vetor $Q \times 1$ de parâmetros de dispersão. Além disso, $\mathcal{Y} = (\boldsymbol{Y}_1^\top, \ldots, \boldsymbol{Y}_R^\top)^\top$ denota o vetor empilhado de ordem $NR \times 1$ da matriz de variáveis resposta $\boldsymbol{Y}_{N \times R}$ e $\mathcal{M} = (\boldsymbol{\mu}_1^\top, \ldots, \boldsymbol{\mu}_R^\top)^\top$ denota o vetor empilhado de ordem $NR \times 1$ da matriz de valores esperados $\boldsymbol{M}_{N \times R}$.

Para estimação dos parâmetros de regressão é utilizada a função quasi-score \citep{Liang86}, representada por
\begin{equation}
\label{eq:qs}
      \begin{aligned}
        \psi_{\boldsymbol{\beta}}(\boldsymbol{\beta},
          \boldsymbol{\lambda}) = \boldsymbol{D}^\top
            \boldsymbol{C}^{-1}(\mathcal{Y} - \mathcal{M}),
\end{aligned}
\end{equation}
\noindent em que $\boldsymbol{D} = \nabla_{\boldsymbol{\beta}} \mathcal{M}$ 
é uma matriz $NR \times K$, e $\nabla_{\boldsymbol{\beta}}$ denota o 
operador gradiente. Utilizando a função quasi-score a matriz $K \times K$
de sensitividade de $\psi_{\boldsymbol{\beta}}$ é dada por
\begin{equation}
\begin{aligned}
S_{\boldsymbol{\beta}} = E(\nabla_{\boldsymbol{\beta} \psi \boldsymbol{\beta}}) = -\boldsymbol{D}^{\top} \boldsymbol{C}^{-1} \boldsymbol{D},
\end{aligned}
\end{equation}
\noindent enquanto que a matriz $K \times K$ de variabilidade de $\psi_{\boldsymbol{\beta}}$ é escrita como
\begin{equation}
\begin{aligned}
V_{\boldsymbol{\beta}} = VAR(\psi \boldsymbol{\beta}) = \boldsymbol{D}^{\top} \boldsymbol{C}^{-1} \boldsymbol{D}.
\end{aligned}
\end{equation}

Para os parâmetros de dispersão é utilizada a função de estimação de
Pearson, definida da forma
    \begin{equation}
    \label{eq:pearson}
      \begin{aligned}
        \psi_{\boldsymbol{\lambda}_i}(\boldsymbol{\beta},
        \boldsymbol{\lambda}) =
        \mathrm{tr}(W_{\boldsymbol{\lambda}i}
          (\boldsymbol{r}^\top\boldsymbol{r} -
          \boldsymbol{C})), \: \: i = 1,.., Q, 
    \end{aligned}
\end{equation}
\noindent em que $W_{\boldsymbol{\lambda}i} = -\frac{\partial
    \boldsymbol{C}^{-1}}{\partial \boldsymbol{\lambda}_i}$ e
    $\boldsymbol{r} = (\mathcal{Y} - \mathcal{M})$. A entrada $(i,j)$ da matriz de sensitividade $Q \times Q$ de $\psi_{\boldsymbol{\lambda}}$ é
dada por
\begin{equation}
      \begin{aligned}
S_{\boldsymbol{\lambda_{ij}}} = E \left (\frac{\partial }{\partial \boldsymbol{\lambda_{i}}} \psi \boldsymbol{\lambda_{j}}\right) = -tr(W_{\boldsymbol{\lambda_{i}}} CW_{\boldsymbol{\lambda_{J}}} C).
    \end{aligned}
\end{equation}
\noindent Já a entrada $(i,j)$ da matriz de variabilidade $Q \times Q$ de $\psi_{\boldsymbol{\lambda}}$ é definida por
\begin{equation}
      \begin{aligned}
V_{\boldsymbol{\lambda_{ij}}} = Cov\left ( \psi_{\boldsymbol{\lambda_{i}}}, \psi_{\boldsymbol{\lambda_{j}}} \right) = 2tr(W_{\boldsymbol{\lambda_{i}}} CW_{\boldsymbol{\lambda_{J}}} C) + \sum_{l=1}^{NR} k_{l}^{(4)} (W_{\boldsymbol{\lambda_{i}}})_{ll} (W_{\boldsymbol{\lambda_{j}}})_{ll},
    \end{aligned}
\end{equation}
\noindent em que $k_{l}^{(4)}$ denota a quarta cumulante de $\mathcal{Y}_{l}$. No processo de estimação dos McGLMs é usada sua versão empírica.

Para se levar em conta a covariância entre os vetores $\boldsymbol{\beta}$
e $\boldsymbol{\lambda}$, \citet{Bonat16} obtiveram as matrizes de 
sensitividade e variabilidade cruzadas, denotadas por $S_{\boldsymbol{\lambda \beta}}$, $S_{\boldsymbol{\beta \lambda}}$ e $V_{\boldsymbol{\lambda \beta}}$, mais detalhes em \citet{Bonat16}. As matrizes de sensitividade e variabilidade conjuntas de $\psi_{\boldsymbol{\beta}}$ e $\psi_{\boldsymbol{\lambda}}$ são denotados por

\begin{equation}
      \begin{aligned}
S_{\boldsymbol{\theta}} = \begin{bmatrix}
S_{\boldsymbol{\beta}} & S_{\boldsymbol{\beta\lambda}} \\ 
S_{\boldsymbol{\lambda\beta}} & S_{\boldsymbol{\lambda}} 
\end{bmatrix} \text{e } V_{\boldsymbol{\theta}} = \begin{bmatrix}
V_{\boldsymbol{\beta}} & V^{\top}_{\boldsymbol{\lambda\beta}} \\ 
V_{\boldsymbol{\lambda\beta}} & V_{\boldsymbol{\lambda}} 
\end{bmatrix}.
\end{aligned}
\end{equation}

Seja $\boldsymbol{\hat{\theta}} = (\boldsymbol{\hat{\beta}^{\top}}, \boldsymbol{\hat{\lambda}^{\top}})^{\top}$ o estimador baseado na \autoref{eq:qs} e \autoref{eq:pearson}, a distribuição assintótica de $\boldsymbol{\hat{\theta}}$ é

\begin{equation}
\begin{aligned}
\boldsymbol{\hat{\theta}} \sim N(\boldsymbol{\theta}, J_{\boldsymbol{\theta}}^{-1}),
\end{aligned}
\end{equation}
\noindent em que $J_{\boldsymbol{\theta}}^{-1}$ é a inversa da matriz de informação de Godambe, dada por
$J_{\boldsymbol{\theta}}^{-1} = S_{\boldsymbol{\theta}}^{-1} V_{\boldsymbol{\theta}} S_{\boldsymbol{\theta}}^{-\top}$, em que $S_{\boldsymbol{\theta}}^{-\top} = (S_{\boldsymbol{\theta}}^{-1})^{\top}.$

Para resolver o sistema de equações $\psi_{\boldsymbol{\beta}} = 0$ e $\psi_{\boldsymbol{\lambda}} = 0$ faz-se uso do algoritmo Chaser modificado, proposto por \citet{jorg04}, que fica definido como

\begin{equation}
\begin{aligned}
\begin{matrix}
\boldsymbol{\beta}^{(i+1)} = \boldsymbol{\beta}^{(i)}- S_{\boldsymbol{\beta}}^{-1} \psi \boldsymbol{\beta} (\boldsymbol{\beta}^{(i)}, \boldsymbol{\lambda}^{(i)}), \\ 
\boldsymbol{\lambda}^{(i+1)} = \boldsymbol{\lambda}^{(i)}\alpha S_{\boldsymbol{\lambda}}^{-1} \psi \boldsymbol{\lambda} (\boldsymbol{\beta}^{(i+1)}, \boldsymbol{\lambda}^{(i)}).
\end{matrix}
\end{aligned}
\end{equation}

Toda metodologia do McGLM está implementada no pacote \emph{mcglm} \citep{mcglm} do software estatístico R \citep{softwareR}.

\section{Testes de hipóteses}

A palavra "inferir" significa tirar conclusão. O campo de estudo chamado de inferência estatística tem como objetivo o desenvolvimento e discussão de métodos e procedimentos que permitem, com certo grau de confiança, fazer afirmações sobre uma popupalação com base em informação amostral. Na prática, costuma ser inviável trabalhar com uma população. Assim, a alternativa usada é coletar uma amostra e utilizar esta amostra para tirar conclusões. Neste sentido, a inferência estatística fornece ferramentas para estudar quantidades populacionais (parâmetros) por meio de estimativas destas quantidades obtidas através da amostra.

Contudo, é importante notar que diferentes amostras podem fornecer diferentes resultados. Por exemplo, se há interesse em estudar a média de determinada característica na população mas não há condições de se observar a característica em todas as unidades, usa-se uma amostra. E é totalmente plausível que diferentes amostras apresentem médias amostrais diferentes. Portanto, os métodos de inferência estatística sempre apresentarão determinado grau de incerteza. 

Campos importantes da inferência estatística são a estimação de quantidades (por ponto e intervalo) e testes de hipóteses. O objetivo desta revisão é apresentar uma visão geral a respeito de testes de hipóteses estatísticas e os principais componentes. Mais sobre inferência estatística pode ser visto em \citet{barndorff2017}, \citet{silvey2017}, \citet{azzalini2017}, \citet{wasserman2013all}, entre outros.

\subsection{Elementos de um teste de hipóteses}

A atual teoria dos testes de hipóteses é resultado da combinação de  trabalhos conduzidos predominantemente na década de 1920 por Ronald Fisher, Jerzy Neyman e Egon Pearson em publicações como \citet{fisherarrangement}, \citet{fisher1929}, \citet{neyman2020use1}, \citet{neyman2020use2} e \citet{neyman1933ix}. 

Entende-se por hipótese estatística uma afirmação a respeito de um ou 
mais parâmetros (desconhecidos) que são estimados com base em uma amostra. Já um teste de hipóteses é o procedimento que permite responder perguntas como: com base na evidência amostral, podemos considerar que dado parâmetro é igual a determinado valor? Alguns dos componentes de um teste de hipóteses são: as hipóteses, a estatística de teste, a distribuição da estatística de teste, o nível de significância, o poder do teste, a região crítica e o p-valor.

Para definição dos elementos necessários para condução de um teste de hipóteses, considere que uma amostra foi tomada com o intuito de estudar determinada característica de uma população. Considere $\hat{\theta}$ a estimativa de um parâmetro $\theta$ da população. Neste contexto, uma hipótese estatística é uma afirmação a respeito do valor do parâmetro $\theta$ que é estudado através da estimativa $\hat{\theta}$ a fim de concluir algo sobre a população de interesse.

Na prática, sempre são definidas duas hipóteses de interesse. A primeira 
delas é chamada de hipótese nula ($H_0$) e trata-se da hipótese de que
o valor de um parâmetro populacional é igual a algum valor especificado. A segunda hipótese é chamada de hipótese alternativa ($H_1$) e trata-se da hipótese de que o parâmetro tem um valor diferente daquele especificado na hipótese nula. Deste modo, através do estudo da quantidade $\hat{\theta}$ verificamos a plausibilidade de se afirmar que $\theta$ é igual a um valor $\theta_0$. Portanto, três tipos de hipóteses podem ser especificadas:

\begin{enumerate}

  \item $H_0: \theta = \theta_0 \, \, vs \, \, H_1: \theta \neq \theta_0$.
  
  \item $H_0: \theta = \theta_0 \, \, vs \, \, H_1: \theta >  \theta_0$.
  
  \item $H_0: \theta = \theta_0 \, \, vs \, \, H_1: \theta < \theta_0$.
  
\end{enumerate}

Com as hipóteses definidas, dois resultados são possíveis em termos de $H_0$: rejeição ou não rejeição. O uso do termo "aceitar" a hipótese nula não é recomendado tendo em vista que a decisão a favor ou contra a hipótese se dá por meio de informação amostral. Ainda, por se tratar de um procedimento baseado em informação amostral, existe um risco associado a decisões equivocadas. Os possíveis desfechos de um teste de hipóteses estão descritos na \autoref{tab:desfechos}, que mostra que existem dois casos nos quais toma-se uma decisão equivocada. Em uma delas rejeita-se uma hipótese nula  verdadeira (erro do tipo I) e na outra não rejeita-se uma hipótese nula falsa (erro do tipo II). 

A probabilidade do erro do tipo I é denotado por $\alpha$ e chamada de nível de significância, já a probabilidade do erro do tipo II é denotado por $\beta$. O cenário ideal é aquele que minimiza tanto $\alpha$ quanto $\beta$, contudo, em geral, à medida que $\alpha$ reduz, $\beta$ tende a aumentar. Por este motivo busca-se controlar o erro do tipo I. Além disso temos que a probabilidade de se rejeitar a hipótese nula quando a hipótese alternativa é verdadeira (rejeitar corretamente $H_0$) recebe o nome de poder do teste.

\begin{table}[h]
\centering
\begin{tabular}{l|cc}
\hline
\multicolumn{1}{c|}{}    & \textbf{Rejeita $H_0$} & \textbf{Não Rejeita $H_0$} \\ \hline
\textbf{$H_0$ verdadeira} & Erro tipo I           & Decisão correta           \\
\textbf{$H_0$ falsa}      & Decisão correta       & Erro tipo II              \\ \hline
\end{tabular}
\caption{Desfechos possíveis em um teste de hipóteses}
\label{tab:desfechos}
\end{table}

A decisão acerca da rejeição ou não rejeição de $H_0$ se dá por meio da avaliação de uma estatística de teste, uma região crítica e um valor crítico. A estatística de teste é o valor utilizado para se tomar a decisão de rejeitar ou não rejeitar a hipótese nula. Trata-se de uma quantidade obtida através de operações da estimativa do parâmetro. Esta estatística segue uma distribuição de probabilidade e esta distribuição é usada para definir a região e o valor crítico.

Considerando a distribuição da estatística de teste, a região crítica consiste do conjunto de valores nos quais rejeita-se a hipótese nula. Já o valor crítico é o valor que divide a área de rejeição da área de não rejeição de $H_0$. Caso a estatística de teste esteja dentro da região crítica, significa que as evidências amostrais apontam para a rejeição de $H_0$. Por outro lado, se a estatística de teste estiver fora da região crítica, quer dizer que os dados apontam para uma não rejeição de $H_0$. O já mencionado nível de significância ($\alpha$) tem importante papel no processo, pois trata-se de um valor fixado e, reduzindo o nível de significância, torna-se cada vez mais difícil rejeitar a hipótese nula.

O último conceito importante para compreensão do procedimento geral de  testes de hipóteses é chamado de nível descritivo, p-valor ou ainda $\alpha^*$. Basicamente, trata-se da probabilidade de a estatística de teste tomar um valor igual ou mais extremo do que aquele que foi observado, supondo que a hipótese nula é verdadeira. Deste modo, o p-valor pode ser visto como uma quantidade que fornece informação quanto ao grau que os dados vão contra a hipótese nula. Esta quantidade pode ainda ser utilizada como parte da regra decisão, uma vez que um p-valor menor que o nível de significância sugere que há evidência nos dados em favor da rejeição da hipótese nula.

Assim, o procedimento geral para condução de um teste de hipóteses 
consiste em: 

\begin{enumerate}
  
  \item Definir $H_0$ e $H_1$.
  
  \item Identificar o teste a ser efetuado, sua estatística de teste e 
distribuição.
  
  \item Obter as quantidades necessárias para o cálculo da estatística de teste.
  
  \item Fixar o nível de significância.
  
  \item Definir o valor e a região crítica.
  
  \item Confrontar o valor e região crítica com a estatística de teste.
  
  \item Obter o p-valor.
  
  \item Concluir pela rejeição ou não rejeição da hipótese nula.
  
\end{enumerate}

\subsection{Testes de hipótese em modelos de regressão}

\textbf{contexto geral}

\textbf{Trv}

\textbf{Wald} 

\textbf{Escore}

\textbf{Visao geral dos 3}

\textbf{Outras possibilidades}


\section{ANOVA e MANOVA}



