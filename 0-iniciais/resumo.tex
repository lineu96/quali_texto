\begin{resumo}

Ciência de dados é um campo de estudo interdisciplinar que compreende áreas como estatística, ciência da computação e matemática. Neste contexto, métodos estatísticos são de fundamental importância sendo que, dentre as possíveis técnicas disponíveis para análise de dados, os modelos de regressão tem papel importante. Tais modelos são indicados a problemas nos quais existe interesse em verificar a associação entre uma ou mais variáveis respostas e um conjunto de variáveis explicativas; isto é feito através da obtenção de uma equação que explique a relação entre as variáveis explicativas e a(s) resposta(s). Existem modelos uni e multivariados: nos modelos univariados há apenas uma variável resposta; já em modelos multivariados há mais de uma resposta. Dentre as classes de modelos multivariados estão os modelos multivariados de covariância linear generalizada (McGLMs). Trata-se de uma flexível clase que permite lidar com múltiplas respostas de diferentes naturezas, correlacionadas entre si em que é possível modelar também a correlação entre indivíduos do conjunto de dados. No contexto de modelos de regressão, um interesse comum costuma ser verificar se a retirada de determinada variável explicativa gera um modelo significativamente pior, ou seja, avalia-se se há evidência suficiente nos dados para afirmar que determinada variável explicativa não possui efeito sobre a resposta. Tais conjecturas são avaliadas através dos chamados testes de hipóteses. Três testes de hipóteses são comuns em regressão: o teste da razão de verossimilhanças, o teste Wald e o teste do multiplicador de lagrange, também conhecido como teste escore. Existem ainda técnicas baseadas em testes de hipóteses tais como a análise de variância (ANOVA) em que o objetivo é a avaliação do efeito de cada uma das variáveis explicativas sobre a(s) resposta(s); isto é feito através da comparação via testes de hipóteses entre modelos com e sem cada uma das variáveis explicativas. Para o caso multivariado estende-se a técnica de análise de variância (ANOVA) para a análise de variância  multivariada (MANOVA). No entanto, considerando os modelos multivariados de covariância linear generalizada, não há discussão a respeito da construção destes testes para a classe. Assim, por se tratar de uma classe de modelos flexível e com alto poder de aplicação a problemas práticos, nosso objetivo geral é o desenvolvimento de testes de hipóteses para os McGLMs. Nossa proposta é adaptar o teste Wald para a realização de testes de hipóteses gerais sobre parâmetros de McGLMs. Temos como objetivos implementar funções para efetuar tais testes, bem como funções para efetuar ANOVAs e MANOVAs. As propriedades e comportamento dos testes propostos serão verificados com base em estudos de simulação e o potencial de aplicação das metodologias discutidas será apresentado com base na aplicação a conjuntos de dados reais.


\end{resumo}
