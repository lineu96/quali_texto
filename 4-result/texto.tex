
\chapter{Resultados preliminares, pendências e cronograma}

Este capítulo é destinado à apresentação das funções já implementadas, apresentação das tarefas a serem cumpridas até o fim do mestrado a cronograma alvo a ser seguido até a defesa.

\label{cap:result}

\section{Funções implementadas}

No capítulo \autoref{cap:proposta} vimos como chegar a um procedimento para realização de testes de hipóteses sobre qualquer parâmetro ou combinação de parâmetros de um McGLM. Deste modo um dos objetivos deste trabalho consiste em implementar tais testes no software R \citep{softwareR} com o intuito de complementar as já possíveis análises permitidas pelo pacote \emph{mcglm} \citep{mcglm}.

No que diz respeito à implementações do teste Wald em outros contextos no R, o pacote \emph{lmtest} \citep{lmtest} possui uma função genérica para realizar testes de Wald para comparar modelos lineares e lineares generalizados aninhados. Já o pacote \emph{survey} \citep{survey1}; \citep{survey2};\citep{survey3} possui uma função que realiza teste de Wald que, por padrão, testa se todos os coeficientes associados a um determinado termo de regressão são zero, mas é possível especificar hipóteses com outros valores. 

O pacote \emph{car} \citep{car} possui uma implementação para testar hipóteses lineares sobre parâmetros de modelos lineares, modelos lineares generalizados, modelos lineares multivariados, modelos de efeitos mistos, dentre outros; nesta implementação o usuário tem total controle de que parâmetros testar e com quais valores confrontar na hipótese nula. 

Quanto às tabelas de análise de variância, o R possui a função anova no pacote padrão \emph{stats} \citep{softwareR} aplicável a modelos lineares e lineares generalizados. Já o pacote \emph{car} \citep{car} possui uma função que retorna quadros de análise de variância dos tipos II e III para diversos modelos. 

Contudo, quando se trata de modelos multivariados de covariância linear generalizada ajustados no pacote \emph{mcglm}, não existem opções para realização de testes de hipóteses lineares gerais nem de análises de variância utilizando a estatística de Wald. 

Deste modo, baseando-nos nas funcionalidades do pacote \emph{car} \citep{car}, implementamos funções que permitem a realização de análises de variância por variável resposta (ANOVA), bem como análises de variância multivariadas (MANOVA). Estas funções recebem como argumento apenas o objeto que armazena o modelo devidamente ajustado através da função \emph{mcglm()} do pacote \emph{mcglm}. Foram implementadas também funções que geram quadros como os de análise de variância focados no preditor linear matricial, ou seja, quadros cujo objetivo é verificar a significância dos parâmetros de dispersão. 

Por fim, foi implementada uma função para hipóteses lineares gerais especificadas pelo usuário, na qual é possível testar hipóteses sobre parâmetros de regressão, dispersão ou potência. Também é possível especificar hipóteses sobre múltiplos parâmetros e o vetor de valores da hipótese nula é definido pelo usuário. Esta função recebe como argumentos o modelo, um vetor com os parâmetros que devem ser testados e o vetor com os valores sob hipótese nula. Com algum trabalho, através da função de hipóteses lineares gerais, é possível replicar os resultados obtidos pelas funções de análise de variância.

Todas as funções geram resultados mostrando graus de liberdade e p-valores baseados no teste Wald aplicado aos modelos multivariados de covariância linear generalizada (McGLM). A \autoref{tab:funcoes} mostra os nomes e descrições das funções implementadas.

\begin{table}[h]
\centering
\begin{tabular}{ll}
\hline
Função                   & Descrição \\ 
\hline

mc\_linear\_hypothesis() & Hipóteses lineares gerais especificadas pelo usuário \\

mc\_anova\_I()           & ANOVA  tipo I \\
mc\_anova\_II()          & ANOVA  tipo II \\
mc\_anova\_III()         & ANOVA  tipo III \\

mc\_manova\_I()          & MANOVA tipo I \\
mc\_manova\_II()         & MANOVA tipo II \\
mc\_manova\_III()        & MANOVA tipo III \\

mc\_anova\_disp()        & ANOVA  tipo III para dispersão \\
mc\_manova\_disp()       & MANOVA tipo III para dispersão \\

\hline
\end{tabular}
\caption{Funções implementadas}
\label{tab:funcoes}
\end{table}

A função \emph{mc\_linear\_hypothesis()} é a implementação computacional em R que permite a execução de qualquer um dos testes apresentados no \autoref{cap:proposta}. É a função mais flexível que temos no conjunto de implementações. Com ela é possível especificar qualquer tipo de hipótese sobre parâmetros de regressão, dispersão ou potência de um modelo \emph{mcglm}. 

As funções \emph{mc\_anova\_I()}, \emph{mc\_anova\_II()} e \emph{mc\_anova\_III()} são funções destinadas à avaliação dos parâmetros de regressão do modelo. Elas geram quadros de análise de variância por resposta para um modelo \emph{mcglm}. As funções \emph{mc\_manova\_I()}, \emph{mc\_manova\_II()} e \emph{mc\_manova\_III()} também são funções destinadas à avaliação dos parâmetros de regressão do modelo. Elas geram quadros de análise de variância multivariada para um modelo \emph{mcglm}. Enquanto as funções de análise de variância simples visam avaliar o efeito das variáveis para cada resposta, as multivariadas visam avaliar o efeito das variáveis explicativas em todas as variáveis resposta simultaneamente. As nomenclaturas seguem o que foi exposto no \autoref{cap:proposta}.

Tal como descrito no \autoref{cap:literatura}, a matriz $\boldsymbol{\Omega({\tau})}$ tem como objetivo modelar a correlação existente entre linhas do conjunto de dados através do chamado preditor linear matricial. Na prática temos, para cada matriz do preditor matricial, um parâmetro de dispersão $\tau_d$. De modo análogo ao que é feito para o preditor de média, podemos usar estes parâmetros para avaliar o efeito das unidades correlacionadas no estudo. Neste sentido implementamos as funções \emph{mc\_anova\_disp()} e \emph{mc\_manova\_disp()}. 

A função \emph{mc\_anova\_disp()} efetua uma análise de variância do tipo III para os parâmetros de dispersão do modelo. Tal como as demais funções com prefixo \emph{mc\_anova} é gerado um quadro para cada variável resposta, isto é, nos casos mais gerais avaliamos se há evidência que nos permita afirmar que determinado parâmetro de dispersão é igual a 0, ou seja, se existe efeito das medidas repetidas tal como especificado no preditor matricial para aquela resposta. Já a função \emph{mc\_manova\_disp()} pode ser utilizada em um modelo multivariado em que os preditores matriciais são iguais para todas as respostas e há o interesse em avaliar se o efeito das medidas correlocionadas é o mesmo para todas as respostas.

Por fim, ressaltamos que as todas as funções de prefixo \emph{mc\_anova} e \emph{mc\_manova} foram implementadas no sentido de facilitar o procedimento de análise da importâncias das variáveis. Contudo, dentre as funções implementadas, a mais flexível é a função \emph{mc\_linear\_hypothesis()} que implementa e da liberdade ao usuário de efetuar qualquer teste utilizando a estatística de Wald no contexto dos McGLMs. A partir desta função é possível replicar os resultados de qualquer uma das funções de análise de variância e testar hipóteses mais gerais como igualdade de efeitos, formular hipóteses com testes usando valores diferentes de zero e até mesmo formular hipóteses que combinem parâmetros de regressão, dispersão e potência quando houver alguma necessidade prática.

\section{Pendências}

Através da adaptação do teste Wald para o contexto dos McGLMs gostaríamos ainda de propor e implementar neste trabalho procedimentos para realização de testes de comparações múltiplas. Tais procedimentos são utilizados quando a análise de variância aponta como conclusão a existência de efeito significativo dos parâmetros associados a uma variável categórica, ou seja, há ao menos uma diferença significativa entre os níveis do fator. Com isso, o teste de comparações múltiplas é utilizado para determinar onde estão estas diferenças. Por exemplo, suponha que há no modelo uma variável categórica $x$ de três níveis: A, B e C. A análise de variância mostrará se há efeito da variável $x$ no modelo, isto é, se os valores da resposta estão associados aos níveis de $x$, contudo este resultado não nos mostrará se os valores da resposta diferem de A para B, ou de A para C, ou ainda se B difere de C. Com isso, testes de comparações múltiplas se encaixam no escopo deste trabalho no sentido de fornecer ferramentas para melhor compreender o efeito das variáveis explicativas sobre as respostas no contexto dos McGLMs.

Ainda nesta linha, outra tarefa a ser cumprida é a adequação dos testes para que sejam válidos para diferentes contrastes. Um contraste é uma combinação linear de variáveis em que a soma dos coeficientes é igual a zero, o que permite a comparação entre os níveis de uma variável categórica. Através da definição dos contrastes é possível estabelecer diferentes comparações entre os níveis de variáveis categóricas que entram no modelo como variáveis explicativas. Logo, os contrastes definem como as variáveis categóricas são tratadas nos modelos. O contraste mais comum utilizado é chamado constraste de tramento em que o primeiro nível da variável categórica é mantido como valor de referência e, para os demais níveis, mede-se a mudança para a categoria de referência; nossa adaptação e funções funcionam para este caso. Para outros tipos de contrastes as funções necessitam de modificações. Tais modificações se encaixam no escopo do trabalho pois a definição dos contrastes define quais as comparações são de interesse em um contexto prático e nem sempre o esquema de constrastes tradicionais se encaixam ao problema.

Com estas modificações feitas, pretendemos avaliar as propriedades e comportamento dos testes propostos com base em estudos de simulação. O objetivo consiste em verificar o poder dos testes sob diferentes configurações de modelos, isto é, variando o número de respostas do modelo, distribuições da(s) resposta(s), tamanhos amostrais, correlação entre respostas, correlação entre unidades do conjunto de dados, dentre outras características a fim de verificar para quais cenários nossa proposta apresenta resultados satisfatórios, para quais cenários os resultados não são satisfatórios de que forma pode-se melhorar o desempenho da proposta nestes casos.

Por fim, visamos finalizar uma dissertação de mestrado que apresente a adaptação do teste, implementação das funções, resultados de estudos de simulação que comprovem o desempenho destas funções e ainda motivar o potencial de aplicação das metodologias discutidas com base na aplicação a conjuntos de dados reais. Considerando o grupo de pesquisa (Data Science \& Big Data) o trabalho teria as seguintes contribuições: propor um teste para uma classe de modelos não usual mas com alto potencial de aplicação em contextos práticos, fornecer ferramentas para cientistas de dados que necessitem efetuar testes de hipóteses sobre parâmetros de modelos de regressão multivariados com o objetivo de compreender o impacto de variáveis explicativas sobre variáveis respostas nos mais diversos contextos, comprovar o funcionamento da proposta através de um estudo de simulação e ainda motivar o uso da proposta através de análises de conjuntos de dados reais de diferentes áreas de aplicação.

\section{Cronograma}

A \autoref{tab:crono} mostra o cronograma alvo a ser seguido até a defesa da dissertação para obtenção da titulação.

\begin{table}[H]
\centering
\begin{tabular}{ccc}
\hline
\textbf{Tarefa}                                   & \textbf{Data de início} & \textbf{Data de finalização} \\ \hline
Implementação dos testes de comparações múltiplas & 17/08                   & 17/09                        \\
Adaptação das funções para diferentes contrastes  & 17/09                   & 17/10                        \\
Desenho e execução do estudo de simulação         & 17/10                   & 17/11                        \\
Análise de dados                                  & 17/11                   & 17/12                        \\
Sumarização dos resultados                        & 17/12                   & 17/01                        \\
Entrega e defesa da dissertação                   & 17/01                   & 17/02                        \\ \hline
\end{tabular}
\caption{Cronograma para cumprimento das pendências para titulação.}
\label{tab:crono}
\end{table}