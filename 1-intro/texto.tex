\chapter{Introdução}

\label{cap:intro}

%=====================================================

% A introdução geral do documento pode ser apresentada através das seguintes seções: Desafio, Motivação, Proposta, Contribuição e Organização do documento (especificando o que será tratado em cada um dos capítulos). O Capítulo 1 não contém subseções\footnote{Ver o Capítulo \ref{cap-exemplos} para comentários e exemplos de subseções.}.

Podemos entender como Ciência de Dados o estudo sistemático de conjuntos de dados com o objetivo de gerar conhecimento sobre determinado assunto. Em suma, o objetivo da Ciência de Dados é extrair informação. Seu processo é caracterizado por etapas como a definição do problema, planejamento do estudo, coleta e análise dos dados e, por fim, a interpretação dos resultados.

Trata-se de um campo de estudo extremamente interdisciplinar que envolve técnicas de áreas como Estatística, Ciência da Computação e Matemática. É uma área que vem ganhando destaque nos últimos anos devido a fatores tais como a popularização do uso de dados nas tomadas de decisão em diversos cenários, difusão do uso de grandes bancos de dados, desenvolvimento e propagação de técnicas modernas e eficientes de análise, sem mencionar o desenvolvimento computacional que permitiu a implementação de técnicas mais complexas para solução de problemas e também que mais pessoas tivessem acesso às técnicas e ferramentas necessárias para se analisar dados.

Alguns dos campos de interesse na Ciência de Dados são: métodos de amostragem, mineração de dados, bancos de dados, técnicas de análise exploratória, probabilidade, inferência, otimização, infraestrutura computacional, plataformas de Big Data, modelos estatísticos, dentre outros.

No contexto de modelos estatísticos, existem os chamados modelos de regressão, dentre os quais podemos citar: os modelos lineares, lineares generalizados, aditivos generalizados, de efeitos aleatórios, aditivos generalizados para locação, escala e forma e ainda os multivariados.

Os modelos de regressão são indicados a problemas nos quais temos interesse em verificar a associação entre uma ou mais variáveis resposta e um conjunto de variáveis explicativas; podemos ainda, além de verificar associação, utilizar o modelo para realizar predições para uma população.

Nos casos univariados mais gerais, estes modelos associam uma única variável resposta, também chamada de variável dependente, a uma ou mais variáveis explicativas, conhecidas como variáveis independentes, covariáveis ou preditoras. 

De forma geral, um modelo de regressão é uma expressão matemática que relaciona a média da variável resposta às variáveis preditoras, em que a variável resposta segue uma distribuição de probabilidade condicional às covariáveis e a média é descrita por um preditor linear. 

O caso mais conhecido é o modelo linear normal, no qual um dos pressupostos é de que a variável resposta, condicional às variáveis explicativas, siga distribuição Normal. Todavia, não são raras as situações em que a suposição de normalidade não é atendida. Uma alternativa, por muito tempo adotada, foi buscar uma transformação da variável resposta a fim de atender os pressupostos do modelo, tal como a família de transformações Box-Cox \citep{boxcox64}. Contudo, este tipo de solução leva a dificuldades na interpretação dos resultados.

Neste contexto, a proposta de maior renome para contornar tais restrições foram os Modelos Lineares Generalizados (GLM) propostos por \citet{Nelder72}. Essa classe de modelos permitiu a flexibilização da distribuição da variável resposta de tal modo que esta pertença à família exponencial de distribuições. Em meio aos casos especiais de distribuições possíveis nesta classe de modelos estão a Bernoulli, Binomial, Poisson, Normal, Gama, Normal inversa, entre outras. Trata-se portanto, de uma classe de modelos de regressão univariados para dados de diferentes naturezas, tais como: dados contínuos simétricos e assimétricos, contagens, proporções, assim por diante. Tais características tornam esta classe uma flexível ferramenta de modelagem aplicável a diversos tipos de problema.

Embora as técnicas citadas sejam úteis, há casos em que são coletadas mais de uma resposta por unidade experimental e há o interesse de modelá-las em função de um conjunto de variáveis explicativas. Para problemas com essa estrutura, uma alternativa são os Modelos Lineares Multivariados, nos quais associa-se um conjunto de respostas a uma ou mais covariáveis. Porém, por maior que seja seu potencial de aplicação, essa classe apresenta limitações como a necessidade de normalidade multivariada, homogeneidade das matrizes de variâncias e covariâncias, além de independência entre as observações.

Uma alternativa para solucionar tais limitações são os Modelos Multivariados de Covariância Linear Generalizada (McGLM) propostos por \citet{Bonat16}. Essa classe permite lidar com múltiplas respostas de diferentes naturezas e, de alguma forma, correlacionadas. Além disso, não há nesta classe suposições quanto à independência entre as observações da amostra, pois a correlação entre observações pode ser modelada por um preditor linear matricial que envolve matrizes conhecidas. 

De forma geral, o McGLM é uma estrutura para modelagem de múltiplas respostas, de diferentes naturezas, em que não há necessidade de observações independentes. Estas características tornam o McGLM uma classe flexível ao ponto de ser possível chegar a extensões multivariadas para modelos de medidas repetidas, séries temporais, dados longitudinais, espaciais e espaço-temporais.

Quando trabalhamos com modelos de regressão, por diversas vezes há o interesse em avaliar os parâmetros do modelo. Isto é, verificar se os valores que associam as variáveis explicativas às variáveis respostas são iguais a determinados valores de interesse. Isto é feito através dos chamados testes de hipótese. 

Em geral, existe o interesse em avaliar se há evidência suficiente para afirmar que o parâmetro que associa a variável explicativa à variável resposta é igual a 0, pois, caso esta afirmação seja verdadeira, podemos concluir que a variável explicativa não está associada à variável resposta. Contudo, através dos testes de hipótese podemos avaliar outros valores diferentes de 0.

Para o caso dos modelos lineares tradicionais existem técnicas como a Análise de Variância (ANOVA), na qual o objetivo é analisar o efeito de cada uma das variáveis explicativas, isto é, avaliar se a retirada de cada variável gera perda ao modelo ajustado. Em outras palavras, na Análise de Variância realizamos sucessivos testes de hipótese para verificar se o parâmetro que associa a variável explicativa à variável resposta é igual a 0.

Quando se está na classe de modelos multivariados para dados gaussianos, extende-se o conceito de Análise de Variância (ANOVA) para a Análise de Variância  Multivariada \citep{manova}, a MANOVA. E dentre os testes de hipótese multivariados já discutidos na literatura, destacam-se o $\lambda$ de Wilk's \citep{wilks}, traço de Hotelling-Lawley \citep{lawley} e \citep{hotelling}, traço de Pillai \citep{pillai} e maior raiz de Roy \citep{roy}.

No entanto, considerando o cenário com múltiplas respostas não gaussianas, são escassas as discussões na literatura a respeito de testes de hipótese sobre os parâmetros do modelo. Deste modo, nosso objetivo geral é o desenvolvimento destes testes de hipótese para os Modelos Multivariados de Covariância Linear Generalizada (McGLM) por se tratar de uma classe de modelos flexível e com alto poder de aplicação a problemas práticos em que se fazem necessários tais testes para avaliação do modelo.

Portanto, este trabalho tem os seguintes objetivos específicos:

\begin{enumerate}
  
  \item Adaptar o teste Wald para realização de testes de hipótese gerais sobre parâmetros de Modelos Multivariados de Covariância Linear Generalizada (McGLM).
  
  \item Implementar funções para efetuar tais testes, bem como funções para efetuar Análises de Variância e Análises de Variância Multivariadas para os McGLM.
  
  \item Demonstrar as propriedades e comportamento dos testes propostos com base em estudos de simulação.
  
  \item Demonstrar o potencial de aplicação das metodologias discutidas com base na aplicação a conjuntos de dados reais.
  
\end{enumerate}

Este trabalho está organizado em sete capítulos: na atual seção foi exposto o tema de forma a enfatizar as características dos modelos lineares e testes de hipóteses. O Capítulo 2 é dedicado à revisão bibliográfica da estrutura dos McGLM. No Capítulo 3 é apresentado e discutido o teste Wald no contexto dos McGLM. No capítulo 4 são mostradas as funções implementadas. O Capítulo 5 apresenta o escopo do estudo de simulação para verificar as principais propriedades dos testes propostos. O Capítulo 6 apresenta os conjuntos de dados que serão usados no trabalho com o objetivo de discutir a aplicação do método a conjuntos de dados reais. E, por fim, no Capítulo 7 são apresentados os comentários finais e são discutidos os resultados esperados do estudo.

%=====================================================
