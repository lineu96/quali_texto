\chapter{Introdução}

\label{cap:intro}

%=====================================================

Desde o surgimento do termo \emph{data science} por volta de 1996 \citep{histds} a discussão sobre o tema atrai pesquisadores das mais diversas áreas \citep{cao2016data}. A ciência de dados é vista como um campo de estudo de natureza interdisciplinar que incorpora conhecimento de grandes áreas como estatística, ciência da computação e matemática \citep{ley2018makes}. \citet{weihs2018data} afirmam que a ciência de dados é um campo em muito influenciado por áreas como informática, ciência da computação, matemática, pesquisa operacional, estatística e ciências aplicadas. Em \citep{cao2016data} é dito que ciência de dados engloba técnicas de como: estatística, aprendizado de máquina, gerenciamento de \emph{big data}, dentre outras. 

Alguns dos campos de interesse da ciência de dados são: métodos de amostragem, mineração de dados, bancos de dados, técnicas de análise exploratória, probabilidade, inferência, otimização, infraestrutura computacional, plataformas de \emph{big data}, modelos estatísticos, dentre outros. \citet{weihs2018data} afirmam que os métodos estatísticos são de fundamental importância em grande parte das etapas da ciência de dados. Neste sentido, os modelos de regressão tem papel importante. Tais modelos são indicados a problemas nos quais existe interesse em verificar a associação entre uma ou mais variáveis resposta (também chamadas de variáveis dependentes) e um conjunto de variáveis explicativas (também chamadas de variáveis independentes, covariáveis ou preditoras). 

Para entender minimamente um modelo de regressão, é necessário compreender o conceito de fenômeno aleatório, variável aleatória e distribuição de probabilidade. Um fenômeno aleatório é uma situação na qual diferentes observações podem fornecer diferentes desfechos. Estes fenômenos podem ser descritos por variáveis aleatórias que associam um valor numérico a cada desfecho possível do fenômeno. Os desfechos deste fenômeno podem ser descritos por uma escala que pode ser discreta ou contínua. Uma variável aleatória é considerada discreta quando os possíveis desfechos estão dentro de um conjunto enumerável de valores. Já uma variável aleatória contínua ocorre quando os possíveis resultados estão em um conjunto não enumerável de valores. Na prática existem probabilidades associadas aos valores de uma variável aleatória, e estas probabilidades podem ser descritas através de funções. No caso das variáveis discretas, a função que associa probabilidades aos valores da variável aleatória é chamada de função de probabilidade. No caso das contínuas, esta função é chamada de função densidade de probabilidade.

Existem ainda modelos probabilísticos que buscam descrever as probabilidades de variáveis aleatórias, as chamadas distribuições de probabilidade. Portanto, em problemas práticos, podemos buscar uma distribuição de probabilidades que melhor descreva o fenômeno de interesse. Estas distribuições são descritas por funções e tais funções possuem parâmetros que controlam aspectos da distribuição como escala e forma, tais parâmetros são quantidades desconhecidas estimadas através dos dados. Na análise de regressão busca-se modelar os parâmetros das distribuições de probabilidade como uma função de outras variáveis. Isto é feito através da decomposição do parâmetro da distribuição em outros parâmetros, chamados de parâmetros de regressão, que dependem de variáveis conhecidas e fixas, as variáveis explicativas. 

Assim, o objetivo dos modelos de regressão consiste em obter uma equação que explique a relação entre as variáveis explicativas e o parâmetro de interesse da distribuição de probabilidades selecionada para modelar a variável aleatória. Em geral, o parâmetro de interesse da distribuição de probabilidades modelado em função das variávis explicativas é a média. Fazendo uso da equação resultante do processo de análise de regressão, é possível estudar a importância das variáveis explicativas sobre a resposta e realizar predições da variável resposta com base nos valores observados das variáveis explicativas. 

Em contextos práticos o processo de análise via modelo de regressão parte de um conjunto de dados. Neste contexto, um conjunto de dados é uma representação tabular em que unidades amostrais são representadas nas linhas e seus atributos (variáveis) são representados nas colunas. Pode-se usar um modelo de regressão para, por exemplo, modelar a relação entre a média de uma variável aleatória e um conjunto de variáveis explicativas. Assume-se então que a variável aleatória segue uma distribuição de probabilidades e que o parâmetro de média desta distribuição pode ser descrito por uma combinação linear de parâmetros de regressão associados às variáveis explicativas. Sendo assim, o conhecimento a respeito da influência de uma variável explicativa sobre a resposta vem do estudo das estimativas dos parâmetros de regressão. A obtenção destes parâmetros estimados se dá na chamada etapa de ajuste do modelo, e isto gera a equação da regressão ajustada.

Existem na prática modelos uni e multivariados. Nos modelos univariados há apenas uma variável resposta e temos interesse em avaliar o efeito das variáveis explicativas sobre essa única resposta. No caso dos modelos multivariados há mais de uma resposta e o interesse passa a ser avaliar o efeito dessas variáveis sobre todas as respostas. Existem inúmeras classes de modelos de regressão, mencionaremos neste trabalho três importantes classes: os modelos lineares, os lineares generalizados e os multivariados de covariância linear generalizada. No cenário univariado, durante muitos anos o modelo linear normal \citep{galton} teve papel de destaque no contexto dos modelos de regressão devido principalmente as suas facilidades computacionais. Um dos pressupostos do modelo linear normal é de que a variável resposta, condicional às variáveis explicativas, segue a distribuição normal. Todavia, não são raras as situações em que a suposição de normalidade não é atendida. Uma alternativa, por muito tempo adotada, foi buscar uma transformação da variável resposta a fim de atender os pressupostos do modelo, tal como a família de transformações proposta por \citet{boxcox64}. Contudo, este tipo de solução leva a dificuldades na interpretação dos resultados.

Com o passar o tempo, o avanço computacional permitiu a proposição de modelos mais complexos, que necessitavam de processos iterativos para estimação dos parâmetros \citep{paula}. A proposta de maior renome foram os modelos lineares generalizados (GLM) propostos por \citet{Nelder72}. Essa classe de modelos permitiu a flexibilização da distribuição da variável resposta de tal modo que esta pertença à família exponencial de distribuições. Em meio aos casos especiais de distribuições possíveis nesta classe de modelos estão a Bernoulli, binomial, Poisson, normal, gama, normal inversa, entre outras. Trata-se portanto, de uma classe de modelos de regressão univariados para dados de diferentes naturezas, tais como: dados contínuos simétricos e assimétricos, contagens, assim por diante. Tais características tornam esta classe uma flexível ferramenta de modelagem aplicável a diversos tipos de problema. 

\citet{capacete} fez uso de modelos lineares generalizados em um problema de resposta binária em que o objetivo era avaliar a probabilidade de uso incorreto do sistema de retenção de diferentes capacetes de motociclistas. \citet{euc} avaliou diferentes modelos para dados de contagem na classe dos GLM para modelar o número de sementes de Eucalyptus cloeziana. \citet{pco2} utilizou um modelo linear generalizado com distribuição gama com o objetivo de avaliar que características influenciam os níveis da pressão parcial de dióxido de carbono ($pCO_2$) em lagos localizados ao sul e centro da Noruega. Mais a respeito dos modelos lineares generalizados pode ser visto em \citet{paula} e \citet{cordeiro}.

Embora as técnicas citadas sejam úteis, há casos em que são coletadas mais de uma resposta por unidade experimental e há o interesse de modelá-las em função de um conjunto de variáveis explicativas. Neste cenário surgem os modelos multivariados de covariância linear generalizada (McGLM) propostos por \citet{Bonat16}. Essa classe pode ser vista com uma extensão multivariada dos GLMs que permite lidar com múltiplas respostas de diferentes naturezas e, de alguma forma, correlacionadas. Além disso, não há nesta classe suposições quanto à independência entre as observações, pois a correlação entre observações pode ser modelada por um preditor linear matricial que envolve matrizes conhecidas. Estas características tornam o McGLM uma classe flexível ao ponto de ser possível chegar a extensões multivariadas para modelos de medidas repetidas, séries temporais, dados longitudinais, espaciais e espaço-temporais.

% TESTES DE HIPOTESE

Quando trabalha-se com modelos de regressão, um interesse comum aos analistas é o de verificar se a retirada de determinada variável explicativa do modelo geraria uma perda no ajuste. Ou seja, uma conjectura de interesse é avaliar se há evidência suficiente nos dados para afirmar que determinada variável explicativa não possui efeito sobre a resposta. Isto é feito através dos chamados testes de hipóteses. Testes de hipóteses são ferramentas estatísticas que auxiliam no processo de tomada de decisão sobre valores desconhecidos (parâmetros) estimados por meio de uma amostra (estimativas). Tal procedimento permite verificar se existe evidência nos dados amostrais que apoiem ou não uma hipótese estatística formulada a respeito de um parâmetro. As suposições a respeito de um parâmetro desconhecido estimado com base nos dados são denominadas hipóteses estatísticas, estas hipóteses podem ser rejeitadas ou não rejeitadas com base nos dados. Segundo \citet{lehmann} podemos atribuir a teoria, formalização e filosofia dos testes de hipótese a \citet{neyman1}, \citet{neyman2} e \citet{fisher}. A teoria clássica de testes de hipóteses é apresentada formalmente em \citet{lehmann2}.

No contexto de modelos de regressão, três testes de hipóteses são comuns: o teste da razão de verossimilhanças, o teste Wald e o teste do multiplicador de lagrange, também conhecido como teste escore. \citet{engle} descreve a formulação geral dos três testes. Todos eles são baseados na função de verossimilhança dos modelos. Um modelo de regressão busca encontrar o valor dos parâmetros que associam variáveis explicativas às respostas que maximizam a função de verossimilhança, ou seja, buscam encontrar um conjunto de parâmetros desconhecidos que façam com o que o dado seja provável (verossímil).

O teste da razão de verossimilhanças, inicialmente proposto por \citet{trv}, é efetuado a partir de dois modelos com o objetivo de compará-los. A ideia consiste em obter um modelo com todas as variáveis explicativas e um segundo modelo sem algumas dessas variáveis. O teste é usado para comparar estes modelos através da diferença do logaritmo da função de verossimilhança. Caso essa diferença seja estatísticamente significativa, significa que as variáveis retiradas do modelo completo prejudicam o ajuste. Caso não seja observada diferença entre o modelo completo e o restrito, significa que as variáveis retiradas não geram perda na qualidade e, por este motivo, tais variáveis podem ser descartadas.

Já o teste Wald, proposto por \citet{wald}, requer apenas um modelo ajustado. A ideia consiste em verificar se existe evidência para afirmar que um ou mais parâmetros são iguais a valores postulados. O teste avalia quão longe o valor estimado está do valor postulado. Utilizando o teste Wald é possível formular hipóteses para múltiplos parâmetros, e costuma ser de especial interesse verificar se há evidência que permita afirmar que os parâmetros que associam determinada variável explicativa a variável resposta são iguais a zero. Caso tal hipótese não seja rejeitada, significa que caso estas variáveis sejam retiradas, não existirá perda de qualidade no modelo.

O teste do multiplicador de lagrange ou teste score \citep{score1}, \citep{score2}, \citep{score3}, tal como o teste Wald, requer apenas um modelo ajustado. No caso do teste escore o modelo ajustado não possui o parâmetro de interesse e o que é feito é testar se adicionar esta variável omitida resultará em uma melhora significativa no modelo. Isto é feito com base na inclinação da função de verossimilhança, esta inclinação é usada para estimar a melhoria no modelo caso as variáveis omitidas fossem incluídas.

De certo modo, os três testes podem ser usados para verificar se a retirada de determinada variável do modelo prejudica o ajuste. No caso do teste de razão de verossimilhanças, dois modelos precisam ser ajustados. Já o teste Wald e o escore necessitam de apenas um modelo. Além disso, os testes são assintóticamente equivalentes. Em amostras finitas estes testes podem apresentar resultados diferentes como discutido por \citet{conflict}.

% ANOVA E MANOVA

Para o caso dos modelos lineares tradicionais existem técnicas como a análise de variância (ANOVA), proposta inicialmente por \citet{anova_fisher}. Segundo \citet{anova1}, a ANOVA é um dos métodos estatísticos mais amplamente usados para testar hipóteses e que está presente em praticamente todos os materiais introdutórios de estatística. O objetivo da técnica é a avaliação do efeito de cada uma das variáveis explicativas sobre a resposta. Isto é feito através da comparação via testes de hipóteses entre modelos com e sem cada uma das variáveis explicativas. Logo, tal procedimento permite que seja possível avaliar se a retirada de cada uma das variáveis gera um modelo significativamente pior quando comparado ao modelo com a variável. Para o caso multivariado extende-se a técnica de análise de variância (ANOVA) para a análise de variância  multivariada \citep{manova}, a MANOVA. E dentre os testes de hipóteses multivariados já discutidos na literatura, destacam-se o $\lambda$ de Wilk's \citep{wilks}, traço de Hotelling-Lawley \citep{lawley}, \citep{hotelling}, traço de Pillai \citep{pillai} e maior raiz de Roy \citep{roy}. 

% PROPOSTA

Buscamos até aqui enfatizar a importância dos modelos de regressão no contexto de ciência de dados e sua relevância na análise de problemas práticos. Além disso ressaltamos a importância dos testes de hipótese e também de procedimentos baeados em tais testes para fins de avaliação da importância das variáveis incluídas nos modelos. No entanto, considerando os modelos multivariados de covariância linear generalizada, não há discussão a respeito da construção destes testes para a classe. Assim, por se tratar de uma classe de modelos flexível e com alto poder de aplicação a problemas práticos, nosso objetivo geral é o desenvolvimento de testes de hipóteses para os McGLMs.

Em nosso trabalho buscamos propor uma adaptação do teste de Wald clássico utilizado em modelos lineares para os McGLMs. A construção do teste Wald em modelos tradicionais é baseada nas estimativas de máxima verossimilhança. Contudo a estatística de teste usada não depende da máxima verossimilhança, e sim de um vetor de estimativas dos parâmetros e uma matriz de variância e covariância destes parâmetros. Assim, por mais que os McGLMs não sejam ajustados com base na maximização da função de verossimilhança para obtenção dos parâmetros do modelo, o método de estimação fornece os componentes necessários para a construção do teste. Neste sentido, das três opções clássicas de testes de hipóteses comumente aplicados a problemas de regressão, o teste Wald se torna o mais atrativo pois basta adaptar a estatística de teste.

Nosso trabalho tem os seguintes objetivos específicos: adaptar o teste Wald para realização de testes de hipóteses gerais sobre parâmetros de modelos multivariados de covariância linear generalizada, implementar funções para efetuar tais testes, bem como funções para efetuar análises de variância e análises de variância multivariadas para os McGLMs, avaliar as propriedades e comportamento dos testes propostos com base em estudos de simulação e avaliar o potencial de aplicação das metodologias discutidas com base na aplicação a conjuntos de dados reais.

Este é um projeto de qualificação e está organizado em cinco capítulos: na atual seção foi exposto o tema de forma a enfatizar as características dos modelos de regressão e a utilidade dos testes de hipóteses neste contexto. O Capítulo 2 é dedicado à revisão bibliográfica da estrutura dos McGLMs e testes de hipótese. No Capítulo 3 é apresentada nossa proposta de adaptação do teste Wald para avaliar suposições sobre parâmetros de um McGLM. No capítulo 4 são mostrados os resultados preliminares. E por fim, no Capítulo 5 são discutidas as tarefas a serem cumpridas até o fim do mestrado, desafios, resultados esperados e o cronograma das atividades.

%=====================================================
